%%%%%%%%%%%%%%%%%%%%%%%%%%%%%%%%%%%%%%%%%%%%%%%%%%%%%%%%%%%%%%%%%%%%%%%%%%%%%%%
%%
%% Uppsala Beamer theme example by Frédéric Haziza <daz@it.uu.se>
%%
%% Describing beamerthemeUppsala version 2008/05/15
%%
%% If you have more than three sections or more than three subsections in at least
%% one section, you might want to use the [hideallsubsections] or
%% [hideothersubsections] switch.  In this case, only the current (sub-) section is
%% displayed in the sidebar and not the full overview.
%%
%% Options are:
%% ===========
%% * hideallsubsections, hideothersubsections
%% * nonumbers,totalnumber
%% * withnav, mylogo
%% * grey
%% * noprogressbar
%%
%% For the sidebar layout:
%% * subsectionsattop, sectionpathattop
%% 
%% Removed
%% =======
%% * sidebarshades
%%
%% Tip
%% ===
%% latex this file to see the theme in action
%%
%%%%%%%%%%%%%%%%%%%%%%%%%%%%%%%%%%%%%%%%%%%%%%%%%%%%%%%%%%%%%%%%%%%%%%%%%%%%%%%%

\documentclass{beamer}
%\documentclass[handout]{beamer}
%\documentclass[notes]{beamer}
%\documentclass[trans]{beamer}

\usetheme[hideothersubsections]{Uppsala}

\usepackage{hyperref}
\usepackage{pgf}
\usepackage{tikz}
\usepgflibrary{arrows}
\usepgflibrary{shapes}
\usepackage{array} % needed for \arraybackslash
\usepackage{graphicx}
\usepackage{adjustbox} % for \adjincludegraphics
\usetikzlibrary{%
  arrows,%
  calc,%
  fit,%
  patterns,%
  plotmarks,%
  shapes.geometric,%
  shapes.misc,%
  shapes.symbols,%
  shapes.arrows,%
  shapes.callouts,%
  shapes.multipart,%
  shapes.gates.logic.US,%
  shapes.gates.logic.IEC,%
  er,%
  automata,%
  backgrounds,%
  chains,%
  topaths,%
  trees,%
  petri,%
  mindmap,%
  matrix,%
  calendar,%
  folding,%
  fadings,%
  through,%
  positioning,%
  scopes,%
  decorations.fractals,%
  decorations.shapes,%
  decorations.text,%
  decorations.pathmorphing,%
  decorations.pathreplacing,%
  decorations.footprints,%
  decorations.markings,%
  shadows}

\usepackage[english]{babel} % or whatever
\usepackage[utf8]{inputenc} % or whatever

\usepackage{mathptmx}
\usepackage{mathpartir}
\usepackage{centernot}
\usepackage{helvet}
%\usepackage{courier}

\usepackage[T1]{fontenc}
% Or whatever. Note that the encoding and the font should match. If T1
% does not look nice, try deleting the line with the fontenc.

%% -----------------------------------------------------------
%% MISC. INFORMATION
%% -----------------------------------------------------------

\title{Access Control for Content Distribution Network Assets}
\subtitle{Researching Copy-on-Write solutions}

\author[Lukas Klingsbo | \emph{lukl8671@student.uu.se}] % appears in the footline
{Lukas Klingsbo <\texttt{lukl8671@student.uu.se}>}

\institute[Dept.\ of Information Technology]% appears in the footline
{Department of Information Technology\\Uppsala University}

\date[]
{April 20, 2016}

%% \logo{...}

%% This is only inserted into the PDF information catalog. Can be left out.
\subject{Access Control for Content Distribution Network Assets}

%% -----------------------------------------------------------
%% Extra ``local'' settings
%% -----------------------------------------------------------

% Comment out this, if you do not want the table of contents to pop up at
% the beginning of each (sub)section:
%%\AtBeginSection[]
%%{
%%  \begin{frame}<beamer> % with <beamer> => doesn't appear in handout mode
%%    \frametitle{Outline} %% Put the title you want, or none!
%%    %\tableofcontents[currentsection,currentsubsection]
%%    \tableofcontents[currentsection]
%%  \end{frame}
%%}

%% Unfolds piecewise element with shading.
%% Text appears, shaded, and the audience knows that somehting is coming
%% Note: if you set the number too high, the audience will try to read the 
%% text that now shows up more, and will be disturbed.
%% ``dynamic'' makes elements show gradually more and more.
\setbeamercovered{transparent=5}
%% \setbeamercovered{dynamic=5}

%% -----------------------------------------------------------

\begin{document}

\begin{frame}[plain] %% Gets the frame to fill up the page, no menu/sidebar/footline
  \titlepage
\end{frame}

\begin{frame}
    \frametitle{Outline}
    \setcounter{tocdepth}{2}
    \tableofcontents[]
\end{frame}

\section{Background}

\subsection{Uprise}
\begin{frame}
  \frametitle{\adjincludegraphics[width=0.2\linewidth,valign=t]{uprise}}
  \centerline{\adjincludegraphics[width=0.2\linewidth,valign=t]{ea}}
  \vspace{30pt}
  \centerline{\adjincludegraphics[width=0.2\linewidth,valign=T]{dice}}
  \vspace{10pt}
  \centerline{\adjincludegraphics[width=0.1\linewidth,valign=T]{visceral}}
  \vspace{10pt}
  \centerline{\adjincludegraphics[width=0.2\linewidth,valign=T]{ghost}}
\end{frame}

\subsection{CDN}
\begin{frame}
  \frametitle{Content Distribution Network (CDN)}
  \centerline{\adjincludegraphics[width=0.9\linewidth,valign=T]{cdn}}
\end{frame}

\subsection{BattleBinary}
\begin{frame}
  \frametitle{BattleBinary}
  - A management system for CDN assets\\
  \vspace{1em}
  What Uprise wanted:
  \begin{itemize}
    \item{Possibility to handle private assets}
    \item{Migration to their current technology stack}
    \item{Features like snapshots and branching}
  \end{itemize}
\end{frame}

\subsection{Copy-on-Write}
\begin{frame}
  \frametitle{Copy-on-Write}
  \centerline{\adjincludegraphics[width=0.9\linewidth,valign=T]{cow-cat}}
  \center{\tiny{*IDs are GUIDs in real implementations}}
\end{frame}

\begin{frame}
  \frametitle{Effects of COW}
  \begin{itemize}
    \item{No accidental incremental changes or race conditions}
    \pause
    \item{No locks needed $\rightarrow$ Scalability}
    \pause
    \item{Take snapshots of system in constant time}
    \pause
    \item{Needs some form of garbage collection/awareness}
     %\pause
    %\item{Low disk space and/or RAM usage}
   \end{itemize}
\end{frame}

\subsection{Related Work}
\begin{frame}
  \frametitle{Related Work}
  \begin{itemize}
    \item{Mach kernel}
    \pause
    \item{Btrfs}
    \pause
    \item{Programming Languages}
  \end{itemize}
\end{frame}


\section{Problem}

\subsection{BattleBinary}
\begin{frame}
  \frametitle{BattleBinary}
  Virtual file system to organise CDN assets
  \centerline{\adjincludegraphics[width=0.9\linewidth,valign=T]{battlebinary}}
  \begin{itemize}
    \item{1. Image is uploaded to BattleBinary}
    \pause
    \item{2. Filename + Part of file's hash = Asset Identifier}
    \pause
    \item{3. Upload image to CDN}
    \pause
    \item{4. Image is available to everybody with
             link\\ \url{http://ea.akamaihd.net/cat-f1ee0283b6accd6.jpg}}
  \end{itemize}
\end{frame}

\subsection{Problem}
\begin{frame}
  \frametitle{Problem}
  \begin{itemize}
    \item{Insecure} % Hashes, anybody can access, disassembling of executables
    \pause
    \item{Impractical} % Can only handle the security settings of one file at the time
    \pause
    \item{Wont scale} % Scales quite well, but not to enormous amounts of users
    \pause
    \item{Lacks necessary features like access control and snapshots}
  \end{itemize}
\end{frame}

\subsection{Private Content}
\begin{frame}
  \frametitle{Private Content}
  \centerline{\adjincludegraphics[width=1.0\linewidth,valign=T]{private-content}}
\end{frame}

\subsection{Solutions Idea}
\begin{frame}
  \frametitle{Solutions Idea}
  \centerline{\adjincludegraphics[width=0.9\linewidth,valign=T]{cdn}}
\end{frame}
\begin{frame}
  \frametitle{Example Operation}
  \centerline{\adjincludegraphics[width=0.9\linewidth,valign=T]{tree}}
  \vspace{20pt}
  \centerline{\tiny{BTRFS: The Linux B-Tree Filesystem, O. Rodeh et al}}
\end{frame}

\section{Perius}
\begin{frame}
  \frametitle{Virtual Filesystem}
  \centerline{\adjincludegraphics[width=0.9\linewidth,valign=T]{front}}
\end{frame}

\subsection{Model}
\begin{frame}
  \frametitle{Model}
  \begin{itemize}
    \item{Describes the system that is to be implemented}
    \pause
    \item{Defines the core operations}
    \pause
    \item{Model informally checked through JPF}
    \pause
    \item{Inspired by Biba. et al}
  \end{itemize}
\end{frame}

\begin{frame}
  \frametitle{Model - Sets}
  \begin{center}
      \begin{tabular}{ | l | l | l | l | p{5cm} |}
          \hline
          \textbf{Set} & \textbf{Name} & \textbf{Relation} & \textbf{Type} \\ \hline
          \textbf{C} & Containers & $\textbf{C} \supseteq C \ni c$ & $\textbf{C}=P_{fin}(P_{fin}(C \cup M))$\\ \hline
          \textbf{M} & Content & $\textbf{M} \supseteq M \ni m$ & $\textbf{M}=P_{fin}(M)$\\ \hline
          \textbf{F} & Files & $\textbf{F} \supseteq F \ni f$ & $\textbf{F}=P_{fin}(F)$\\ \hline
      \end{tabular}
  \end{center}
\end{frame}

\begin{frame}
  \frametitle{Model - File Creation}
  Rule of Inference
  \begin{equation} \label{eq:filecreate}
      \inferrule
      {C \ni c \\ A[u,c] \\ m \centernot \in c \\ \neg readOnly(c)} 
      {(C \cup \{c\}, F) \xrightarrow{u, create(m,c)} (C \cup \{c \cup \{m\}\}, F \cup
      \{m.file\})}
  \end{equation}
\end{frame}

\subsection{Implementation}
\begin{frame}
  \frametitle{Persistent Storage}
  \centerline{\adjincludegraphics[width=0.4\linewidth,valign=T]{mongodb}}
  \begin{itemize}
    \item{Unaware of Copy-on-Write}
    \item{JSON (sort of) documents stored instead of tables}
  \end{itemize}
\end{frame}

\begin{frame}
  \frametitle{Stack}
  \centerline{\adjincludegraphics[width=0.9\linewidth,valign=T]{techstack}}
\end{frame}

\begin{frame}
  \frametitle{Security Settings}
  \begin{itemize}
    \item{\textbf{Public} - The content can be reached by anybody}
    \pause
    \item{\textbf{Protected} - The content can only be reached by a range if IP addresses}
    \pause
    \item{\textbf{Private} - The content can only be reached by users with a signed cookie}
  \end{itemize}
\end{frame}

\begin{frame}
  \frametitle{Snapshots (and branching)}
  \centerline{\adjincludegraphics[width=1.1\linewidth,valign=T]{tree}}
\end{frame}

\begin{frame}
  \frametitle{Load Testing}
  \begin{itemize}
    \item{Rigorous Load Testing was done on the back-end}
    \pause
    \item{Wrk and Apache Bench was used}
    \pause
    \item{Result: About 60000 clients/back-end node before congestion}
  \end{itemize}
\end{frame}

\begin{frame}
  \frametitle{Scalability}
  \begin{itemize}
    \item{Unlimited scaling on width}
    \pause
    \item{Optional Non-Blocking I/O with Reactive Mongo}
    \pause
    \item{Some heavy operations can be solved with caching} % Tree building
  \end{itemize}
\end{frame}

\begin{frame}
  \frametitle{Open Source}
  \begin{itemize}
    \item{\url{https://github.com/spydon/perius}}
    \item{\url{https://github.com/spydon/perius-front}}
  \end{itemize}
\end{frame}

\begin{frame}
    \center{Questions?}
\end{frame}

\begin{frame}
    \center{Thank you for listening}
\end{frame}

\end{document}
