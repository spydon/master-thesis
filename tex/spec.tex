\documentclass[a4paper,12pt]{article}
\usepackage{fullpage}
\usepackage[T1]{fontenc}
\usepackage{amsmath}
\usepackage{amssymb}
\usepackage[utf8]{inputenc}
\usepackage{color}
\usepackage{authblk}
\usepackage{todonotes}
\usepackage{caption}
\usepackage{url}
\usepackage{float}
\usepackage{sectsty}
\usepackage{pdfpages}
\usepackage[section]{placeins}
\DeclareCaptionFont{white}{\color{white}}
\DeclareCaptionFormat{listing}{\colorbox{gray}{\parbox{\textwidth}{#1#2#3}}}
\captionsetup[lstlisting]{format=listing,labelfont=white,textfont=white}

\usepackage{setspace}
\usepackage[toc,page]{appendix}
\usepackage{framed}
\usepackage{geometry}

\usepackage{alltt}
\usepackage{subfig}

% Change section fonts
\allsectionsfont{\sffamily}

% For code box
\usepackage{xcolor}
\usepackage{listings}
\usepackage{caption}
\DeclareCaptionFont{white}{\color{white}}
\DeclareCaptionFormat{listing}{%
  \parbox{\textwidth}{\colorbox{gray}{\parbox{\textwidth}{#1#2#3}}\vskip-4pt}}
  \captionsetup[lstlisting]{format=listing,labelfont=white,textfont=white}
  \lstset{frame=lrb,xleftmargin=\fboxsep,xrightmargin=-\fboxsep}
% End code box

\usepackage{cite}

% General parameters, for ALL pages:
\renewcommand{\topfraction}{0.9}	% max fraction of floats at top
\renewcommand{\bottomfraction}{0.8}	% max fraction of floats at bottom
% Parameters for TEXT pages (not float pages):
\setcounter{topnumber}{2}
\setcounter{bottomnumber}{2}
\setcounter{totalnumber}{4} % 2 may work better
\setcounter{dbltopnumber}{2} % for 2-column pages

\addtolength{\topmargin}{0.5in}

\usepackage{fancyvrb}

\usepackage{tikz} \usetikzlibrary{trees}
\usepackage{hyperref} % should always be the last package

% useful colours (use sparingly!):
\newcommand{\blue}[1]{{\color{blue}#1}}
\newcommand{\green}[1]{{\color{green}#1}}
\newcommand{\red}[1]{{\color{red}#1}}

% useful wrappers for algorithmic/Python notation:
\newcommand{\length}[1]{\text{len}(#1)}
\newcommand{\twodots}{\mathinner{\ldotp\ldotp}} % taken from clrscode3e.sty
\newcommand{\Oh}[1]{\mathcal{O}\left(#1\right)}

% useful (wrappers for) math symbols:
\newcommand{\Cardinality}[1]{\left\lvert#1\right\rvert}
\newcommand{\Ceiling}[1]{\left\lceil#1\right\rceil}
\newcommand{\Floor}[1]{\left\lfloor#1\right\rfloor}
\newcommand{\Iff}{\Leftrightarrow}
\newcommand{\Implies}{\Rightarrow}
\newcommand{\Intersect}{\cap}
\newcommand{\Sequence}[1]{\left[#1\right]}
\newcommand{\Set}[1]{\left\{#1\right\}}
\newcommand{\SetComp}[2]{\Set{#1\SuchThat#2}}
\newcommand{\SuchThat}{\mid}
\newcommand{\Tuple}[1]{\langle#1\rangle}
\newcommand{\Union}{\cup}
\usetikzlibrary{positioning,shapes,shadows,arrows}
\providecommand{\keywords}[1]{\textbf{\textit{Keywords: }} #1}

\title{\textbf{Specification: Providing an Access Control Layer in Content Distribution Networks}}
\author{Lukas Klingsbo}

\begin{document}

\maketitle

\setcounter{tocdepth}{3}
\tableofcontents

\clearpage
\pagenumbering{arabic}
\setcounter{page}{1}
\section{Background}
This work will be done at the company Uprise (Dragarbrunnsgatan 36C, Uppsala). 
The goal of this project is to explore, and creating, the possibility to have 
an local access control layer for files in a content distribution network.
This will make it easier to have control over the security settings of the 
projects and will also make it possible to enforce local (as in not on the CDN) 
auth and audit logging.

\section{Description}
The project should research and possibly develop a system where it is possible 
to set different types of access to different files in a CDN. 
For example, the access control could be divided into three groups:
\begin{itemize}
  \item Very secret - Only accessible from white listed IP addresses
  \item Secret - Only accessible from authenticated users with the correct access
  \item Public - Accessible by anybody
\end{itemize}

When a file is set to a belong to one of these groups it should not be static, 
it should be possible to, at a later stage, assign the file to another group.

It should also be possible to create so called views and projects which a group 
of files are added to, which then have their own security settings.

\section{Methods}
\begin{itemize}
  \item The prototype/system will be developed in Scala and React. 
  \item A literature study will be done once the specification is accepted. 
  \item Results will be evaluated from a security and scalability point of view. 
  \item A work station and desk space will be provided by Uprise.
\end{itemize}

\section{Relevant Courses}
\begin{itemize}
  \item Computer Networks I (1DT052)
  \item Computer Networks II (1DT074)
  \item Distributed Systems (1DT064)
  \item Secure Computer Systems I (1DT072)
  \item Secure Computer Systems II (1DT073)
  \item Process Oriented Programming (1DT083)
\end{itemize}

\section{Delimitations}
If the project is easier than expected it could be expanded by looking into 
automatically doing simpler image analysis to find images that are about to 
be assigned to the incorrect category, for example a ``Very secret'' image 
that is about to be published as ``Public''.

It is possible to do delimitation of the core research question as the project 
proceeds, but it will most likely not be needed as the research question is 
quite concise.

\section{Time Plan}
Note that this is just a rough time plan that will change during the course of 
the project when the circumstances change.
There will rarely be one task at a time, but rather a lot of the tasks will be 
overlapping.
\begin{itemize}
  \item A few days - Properly setting up the work station
  \item 2 weeks - Literature study and obtaining relevant books etc
  \item 20 weeks - Writing the report should be done pretty much 
  throughout the whole project
  \item 1 week - Analyze the current system for managing the CDN assets
  \item 1 week - Analyze what has been done earlier regarding the 
  subject and determine if anything is useful for this project
  \item 4 weeks - Implement a prototype
  \item 1 week - Scalability testing
  \item 3 weeks - Security testing
\end{itemize}

\section{References}
One possibility to connect the backend to:

\url{http://docs.aws.amazon.com/AmazonCloudFront/latest/DeveloperGuide/private-content-signed-cookies.html}

\end{document}
